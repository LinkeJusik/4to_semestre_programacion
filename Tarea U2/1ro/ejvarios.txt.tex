\documentclass[12pt]{article}

\usepackage{hyperref}
\usepackage[spanish]{babel}
\usepackage[letterpaper]{geometry}
\usepackage{amsmath}
\usepackage{amssymb}
\usepackage{amsfonts}

\author{Mart\'{i}nez Gonz\'alez, Gabriel}
\date{today}
\title{Documentos de LATEX}

\begin{document}

\maketitle

\begin{abstract}

	De acuerdo a la lista de ejercicios otorgada, para los ejercicios del 1 al 11 no se auxilie de ninugna herramienta de c\'alculo, m\'as que de usted mismo. La entrega debe realizarse en un archivo PDF generado del c\'odigo \LATEX\. El archivo debe tener nombre ejvarios.pdf. Los ejercicios obligatorios son 1, 3 y 11. La fecha límite de entrega es el 21 de febrero de 2022.

\end{abstract}

\section{Transformaciones de bases}

Complete la siguieente tabla. Al inicuio de cada columna se indica la base en que debe estar el \'umero de la celda correspondiente. En cada rengl\'on se tiene el mismo n\'umero, pero representado en distintas bases.

\begin{center}
\begin{tabular}{||l|c|c|c|r||}

\hline
base 2 & base 4 & base 8 & base 10 & base 16 \\ \hline
110010000001 & 302001 & 6201 & 3201 & C81 \\ \hline
11101001010 & 131022 & 3512 & 1866 & 74A \\ \hline
100101010110 & 211112 & 4526 & 2390 & 956 \\ \hline
11001111001 & 121321 & 3171 & 1657 & 679 \\ \hline
100111010010 & 213102 & 4722 & 2514 & 9D2 \\ \hline
11010111000 & 122320 & 3270 & 1720 & 6B8 \\ \hline
101111111000 & 233320 & 5770 & 3064 & BF8 \\ \hline
11001000001 & 121001 & 3101 & 1601 & 641 \\ \hline
101011011000 & 223120 & 5330 & 2776 & AD8 \\ \hline
10011011001 & 103121 & 2331 & 1241 & 4D9 \\ \hline
110010100 & 12110 & 624 & 404 & 194 \\ \hline
110011001010 & 303022 & 6312 & 3274 & CCA \\ \hline
1111010000 & 33100 & 1720 & 976 & 3D0 \\ \hline
11 10 11 10 11 10 & 32 32 32 & 7356 & 3822 & EEE \\ \hline
101001100101 & 221211 & 5145 & 2661 & A65 \\ \hline
10100001001 & 110021 & 2411 & 1289 & 509 \\ \hline
100001010101 & 201111 & 4125 & 2133 & 855 \\ \hline
111001011011 & 321123 & 7133 & 3675 & E5B \\ \hline
110000000100 & 300010 & 6004 & 3076 & C04 \\ \hline
111011100100 & 323210 & 7344 & 3812 & EE4 \\ \hline
110101100000 & 311200 & 6540 & 3424 & D60 \\ \hline
1100101001 & 30221 & 1451 & 809 & 329 \\ \hline
11010011111 & 122133 & 3237 & 1695 & 69F \\ \hline
10010100101 & 102211 & 2245 & 1189 & 4A5 \\ \hline
110000000 & 12000 & 600 & 384 & 180 \\ \hline
10101100000 & 111200 & 2540 & 1376 & 560 \\ \hline
101001110111 & 221313 & 5167 & 2679 & A77 \\ \hline
10001110 & 2032 & 216 & 142 & 8E \\ \hline
101001110101 & 221311 & 5165 & 2677 & A75 \\ \hline
101111000010 & 233002 & 5702 & 3010 & BC2 \\ \hline
110111010 & 12322 & 672 & 442 & 1BA \\ \hline
100001011110 & 201132 & 4136 & 2142 & 85E \\ \hline
10011101011 & 103223 & 2353 & 1259 & 4EB \\ \hline
100101001110 & 211032 & 4516 & 2382 & 94E \\ \hline
1100100 & 1210 & 144 & 100 & 64 \\ \hline
11110110 & 3312 & 365 & 246 & F6 \\ \hline
10111001111 & 113033 & 2717 & 1487 & 5CF \\ \hline
101101101000 & 231220 & 5550 & 2920 & B68 \\ \hline
11111110101 & 133311 & 3765 & 2037 & 7F5 \\ \hline
101111101110 & 233232 & 5756 & 3054 & BEE \\ \hline
111111011101 & 333131 & 7735 & 4061 & FDD \\ \hline
110100010000 & 310100 & 6420 & 3344 & D10 \\ \hline
11010110101 & 122311 & 3265 & 1717 & 6B5 \\ \hline
10101111101 & 111331 & 2575 & 1405 & 57D \\ \hline
10011101101 & 103231 & 2355 & 1261 & 4ED \\ \hline
10000011000 & 100120 & 2030 & 1048 & 418 \\ \hline
110100110000 & 310300 & 6460 & 3376 & D30 \\ \hline
1010010000 & 22100 & 1220 & 656 & 290 \\ \hline
10100101101 & 110231 & 2455 & 1325 & 52D \\ \hline
11110011010 & 132122 & 3632 & 1946 & 79A \\ \hline
100111110101 & 213311 & 4765 & 2549 & 9F5 \\ \hline
111001111101 & 321331 & 7175 & 3709 & E7D \\ \hline
1110100100 & 32210 & 1644 & 932 & 3A4 \\ \hline
11111010100 & 133110 & 3724 & 2004 & 7D4 \\ \hline
110001000100 & 301010 & 6104 & 3140 & C44 \\ \hline
11110001111 & 132033 & 3617 & 1935 & 78F \\ \hline
110111101110 & 313232 & 6756 & 3566 & DEE \\ \hline
10000111100 & 100330 & 2074 & 1084 & 43C \\ \hline
110000110010 & 300302 & 6062 & 3122 & C32 \\ \hline

\end{tabular}
\end{center}

\section{Evaluación de expresiones}

Eval\'ue las siguientes expresiones e indique algún comentario en caso de haber una incosistencia o particularidad del lenguake de programaci\'on en C o en Java. Utilizamos vardouble para indicar una variable de tipo b\'asico o primitivo double y varint para indicar una variable de tipo b\'asico o primitivo int.

\item ( ( ( ( \~ -4 ) || ( -3 == -1 ) ) ) ? ( ( ( ! -1 ) != ( vardouble = 4 ) ) ) : ( ( ( ! 2 ) \&\& (-2 >= 4 ) ) ) \&\& ( ( ( 2 \!= 1 ) ∗ ( varint \&= 3 ) ) << ( ( \~ 1 ) − ( \~ -4 ) ) ) ) \longrightarrow true
\item ( -5 ∗ -2 )\longrightarrow -10 = true
\item ( ( ( ! -3.47 ) > ( \~ 1 ) ) < ( ( 1 >> 3 ) \& ( \~ 1 ) ) ) \longrightarrow false
\item ( ( ( - ( ! -1 ) ) == ( + ( \~ -4 ) ) ) \& ( ( ( ! -3.47 ) ) ? (( 1 \& 3 ) ) : (( ! 2 ) ) || ( ( ! -1.88 ) == ( \~ 1 ) ) ) ) \longrightarrow false
\item ( -5 \& -3 ) \longrightarrow true
\item ( ( ! -1.88 ) \& ( ~ -4 ) ) \longrightarrow false
\item ( ( \~ ( -5 | -4 ) ) || ( ( 4 - -1 ) / ( ! 4 ) ) ) \longrightarrow false Nota: esta expresi\'on depende de algunas excepciones. Sin embargo, en general sería falsa.
\item ( 4 \% 2 ) \longrightarrow 0 = false
\item ( ( ! -1 ) + ( \~ -4 ) ) \longrightarrow true
\item ( ( 4 - -1 ) > ( 2 + 1.93 ) ) \longrightarrow true
\item ( - ( 4 | -3 ) ) \longrightarrow true
\item ( - ( ( ( ! -1.88 ) * ( \~ -4 ) ) >> ( ( -2 \% -5 ) >= ( varint \&= -5 ) ) ) ) \longrightarrow false
\item ( ( ( \~ -4 ) \& ( \~ 1 ) ) - ( ( -1 + -4 ) >= ( 0.57 == -4 ) ) )\longrightarrow true
\item ( ( ! -1 ) << ( 2 \&\& 1 ) )\longrightarrow false
\item ( ( ( ! ( ! 4 ) ) \% ( ( ! -1.88 ) != ( \~ 1 ) ) ) * ( ( ( ! 2 ) \& ( \~ -4 ) ) + ( + ( -5 \& 3.06 ) )) )\longrightarrow acci\'on invalida. No se puede usar un operador binario en una evaluación entre números enteros. La sintaxis correcta es ( ( ( ! ( ! 4 ) ) \% ( ( ! -1.88 ) != ( \~ 1 ) ) ) * ( ( ( ! 2 ) \& ( \~ -4 ) ) + ( + ( -5 \&\& 3.06 ) )) )\longrightarrow acci\'on invalida
\item ( ( vardouble = -3 ) \& ( -1 + -4 ) )\longrightarrow acci\'on invalida. No se puede usar un operador binario en una evaluación entre números enteros. La sintaxis correcta es ( ( vardouble = -3 ) \&\& ( -1 + -4 ) ) \longrightarrow true
\item ( \~ ( ( ( -1.46 \&\& 0 ) != ( ! -1.88 ) )\^( ( ! -1.88 ) != ( ! -1.88 ) ) ) )\longrightarrow true
\item ( 4 >= -1 )\longrightarrow true
\item ( + ( ( ( -1 + -4 ) * ( 4 \% 2 ) ) / ( - ( -2 * 1 ) ) ) )\longrightarrow false
\item ( ( ( ! -1.88 ) / ( \~ -4 ) ) \<\< ( ( ! 2 ) <= ( \~ 1 ) ) )\longrightarrow false
\item ( ( ( ( 1 \&\& -2 ) << ( \~ -4 ) )\^( ( -5 \% 0 ) \% ( \~ 1 ) ) ) != ( ( ( -1 == -2 ) & ( vardouble /= 0 ) ) \& ( \~ ( 2 ∗ 4 ) ) ) )\longrightarrow acci\'on invalida. Nota: hay un error de sintaxis, pues es un determinado punto se \& en lugar de \&\&; pero, adem\'as, hay una divisi\'on entre 0.
\item ( ( ( ! -1.88 ) > ( vardouble /= -4 ) ) \>\> ( ! ( -3 == -1 ) ) )\longrightarrow false
\item ( ( \~ -4 ) | ( \~\  -4 ) )\longrightarrow true
\item ( ! 4 )\longrightarrow false
\item ( ( + ( ( 0 || 2 ) \% ( \~ -4 ) ) ) | ( ( ( \~ 1 ) == ( -1.46 \&\& 0 ) ) ) ? ( ( ( -5 \&\& -2.65 ) ) ? (( ! -1 ) ) : (( \~ -4 ) ) ) : ( ( ( ! 2 ) ) ? (( ! 4 ) ) : (( \~ -4 ) ) ) )\longrightarrow false
\item ( ( 0 \<\< -2.49 ) ) ? (( -2 \&\& 3.35 ) ) : (( -2 \&\& 3.35 ) )\longrightarrow false$ Nuevamente, hay una acción invalida al querer operar de forma binaria con n\'umeros enteros y flotantes.
\item ( -2 * 0 )\longrightarrow 0=false
\item ( ( ( ( -3 << -5 ) \<\< ( vardouble / = 0 ) ) \& ( ( 4 \% 2 ) == ( ! -1 ) ) ) < ( ( ( -5 \& 3.06 ) >= ( -5 ! = 3.01 ) ) >= ( ( -2.86 + 4 ) ! = ( vardouble = 4 ) ) ) )\longrightarrow acci\'on invalida. Hay algunos errores entre operadores binarios y números enteros, así como una división entre 0.
\item ( ( -2 || -3 ) == ( ! 2 ) )\longrightarrow false
\item ( ( 2 >= 3 ) - ( 2 >= 3 ) )\longrightarrow false
\end{enumerate}

\section{D\'{i}a del año}

Si nos dan dos fechas de un mismo a\~no y nos indican en qu\'e d\'{i}a de la semana se supone que cae la primera fecha, ¿en qu\'e d\'ia de la semana deber\'{i}a caer la segunda fecha?

\nocite{bm,asciiwiki,asciitab,isowiki,isotab1,isotab2,unicodewiki,unicodetab1,unicodetab2,unicodetab3,utfwiki,utftab1,utftab2}
\bibliography{Citas} \bibliographystyle{alpha}

\end{document}