\documentclass[12pt]{article}

\usepackage{hyperref}
\usepackage[spanish]{babel}
\usepackage[letterpaper]{geometry}
\usepackage{amsmath}
\usepackage{amssymb}
\usepackage{amsfonts}

\author{Mart\'{\i}nez Gonz\'alez, Gabriel}
\date{\today}
\title{Tarea 4: Acotaciones}

\begin{document}
\maketitle
\begin{abstract}

	En el documento se resolveran los ejercicios concernientes al cuarto documento de tareas referente a las acotaciones de funciones.	
	
\end{abstract}

Definición 1: Suponga que $f,g:N\longrightarrow R^{+}$ y que existen $n_0\in N$ y $C\in R^{+}$ tales que, $\forall n\geq \n_0$ se cumple $f(n)\leq Cg(n)$, entonces decimos que $f=O(g)$.

\section{Parte 1:}

Muestre que si $f,g:N\longrightarrow R^{+}$ y $$\lim\limits_{n\rightarrow\infty}{\frac{f(n)}{g(n)}}=k$$ para $k$ una contante real, entonces $f=O(g)$.\\

Demostración:\\

El hecho de que el límite se cumpla implica que, a partir de cierta $n\geq n_1, {|{\frac{f(n)}{g(n)}-k}|<\varepsilon_0}$, donde $\varepsilon_0$ está en función de $n$ y dado que $g(n)$ es una funci\'on positiva, se cumple que 
$$|f(n)-kg(n)|<\varepsilon_0 g(n)$$ 
Así pues, para $\varepsilon=\varepsilon_0 g(n)$ suficientemente pequeño; es decir, a partir de $n\geq n_0$, tenemos que:
$$f(n)=k(g)$$
Lo cual satisface la definición tal que $f=O(g)$.

\section{Parte 2:}
Para $f,g,f', g', h:\N\longrightarrow\R^{+}$, pruebe que:\\
a) Si $g=O(f)$ y $h=O(f)$, entonces $g+h=O(f)$\\
Prueba: $g=O(f)\Longrightarrow g(n)\leq C_1f(n)$ y $h=O(f)\Longrightarrow h(n)\leq C_2f(n)$; así pues $g(n)+h(n)\leq C_1f(n)+C_2f(n)=(C_1+C_2)f(n)$. Así pues, denotamos $C=C_1+C_2$ y así pues, tenemos que:
$g(n)+h(n)\leq Cf(n)\Longrightarrow g+h=O(f)$, lo cual completa la prueba.\\

b) Si $f=O(g)$ y $g=O(h)$, entonces $f=O(h)$\\
Prueba: $f=O(g)\Longrightarrow f(n)\leq C_1g(n)$ y $g=O(h)\Longrightarrow g(n)\leq C_2h(n)$. Entonces 
$f(n)\leq C_1g(n)\leq C_1(C_2h(n))=Ch(n)$, para $C=C_1C_2$. Entonces, tenemos que $f(n)\leq Ch(n)\Longrightarrow f=O(h)$. Esto completa la prueba. \\

c) $f+g=O(max{f,g})$\\
Prueba: Supongamos $max{f,g}=f$, entonces tenemos que $f(n)+g(n)\leq Cf(n)\Longrightarrow f+g=O(f)$. Dado que escogimos el m\'aximo de las funciones de forma arbitraria, la prueba es igual de valida para $max{f,g}=g$. Entonces, finalmente, tenemos que $f(n)+g(n)\leq Cmax{f(n), g(n)}\Longrightarrow f+g=O(max{f,g})$. Lo cual completa la prueba.\\

d) $O(f+g)=O(max{f,g})$\\
Prueba: Inmediata a partir del inciso c). Basta observar que $f(n)+g(n)\leq C_1max{f(n), g(n)}\Longrightarrow k(f(n)+g(n))\leq kC_1max{f(n), g(n)}\Longrightarrow O(f+g)=O(max{f,g})$, dado que $kC_1$ es una constante, para $k$ y $C_1$ constantes.\\

e) Si $f=O(f')$ y $g=O(g')$, entonces $f+g=O(f'+g')$\\
Prueba: Esta proposici\'on se puede demostrar usando el inciso c), si demostramos que $max{f,g}\leq k(f'+g')$. Esto es inmediato, dado que, por hipotesis $f\leq k_1f'$ y $g\leq k_2g'$, por lo que, sea cual sea el m\'aximo entre $f$ y $g$, $k(f'+g')$ es mayor o igual, definiendo $k=k_1k_2, k_1,k_2>1$. Esto completa la prueba.\\

f) Si $f=O(f')$ y g=o(g'), entonces $fg=O(f'g')$\\
Prueba: $f(n)\leq C_1f'(n), g(n)\leq C_2g'(n)\Longrightarrow f(n)g(n)=C_1C_2f'(n)g'(n)=Cf'(n)g'(n), C=C_1C_2\Longrightarrow fg=O(f'g')$. \\

g) Si $f=O(g)$ y $a>0$, entonces $af=O(g)$\\
Prueba: $f(n)\leq kg(n)\Longrightarrow af(n)\leq akg(n)=Cg(n), C=ak$. Esto completa la prueba.\\

h) Si $f=O(g)$ y $n\geq 1$, entonces $f^{n}=O(g^{n})$\\
Prueba (por inducción matemática):\\
Para $n=1$ la prueba es inmediata. Para $n=2$: $f=O(g)\Longrightarrow f(n)\leq Cg(n)\Longrightarrow f^{2}(n)\leq C^2g^2(n)\Longrightarrow f^2=O(g^2)$.
Así pues, podemos asumir que se cumple para $n$, tal que $$f^n=O(g^n)$$
Finalmente, demostramos que se cumple para $n+1$: \\
Por hipotesis de inducci\'on
$$f^n=O(g^n)\Longrightarrow f^n\geq Cg^n \Longrightarrow f^{n+1}\leq C^2 g^{n+1}$$
Esto completa la prueba.

\section{Partes de la 3 a la 13}
Para cada caso, indique la mejor estimación entre las siguientes:

3. $f(n)=\binom{n}{3}$\\
$\binom{n}{3}=\frac{n!}{3!(n-3)!}=\frac{n!}{6(n-3)!}$. 
Sabemos que $6(n-3)!$ es un entero, por lo que $n!(6(n-3)!)\geq n!\Longrightarrow n! \geq \frac{n!}{6(n-3)!}$
Por lo que la mejor aproximaci\'on es $f(n)=O(n!)$\\

4. $f(n)=10 ln^3(n)+20n^2$\\
Por los incisos c) de la sección 2:
Tenemos que $ln^3(n)\leq n^2, \forall n$ y $n^2\leq n^2, \forall n$; por lo que la mejor aproximaci\'on es f(n)=O(n^2)\\

5. El n\'umero de monomios en $x, y, z$ de grado total a lo m\'as $n$.\\
La mejor aproximaci\'on es $f(n)=O(n^3)$\\


6. El n\'umero de polinomios en $x$ de grado a lo m\'as $n$ cuyos coeficientes son 0 o 1.\\
En este caso tenemos una colección de n coeficientes que pueden ser o bien 0, o 1. Para calcular el número total de estas listas, tendríamos que ver que el primer coeficiente tiene dos posibilidades (ser 0 o 1), el segundo tambi\'en (ser 0 o 1) y as\'{i} susesivamente hasta el coeficiente de grado $n$, donde sólo existe una posibilidad (pues de ser 0, el polinomio sería de grado n-1). Así pues, tenemos que la cantidad de polinomios se puede poner de la forma:
$$2\cdot2\cdot...\cdot2-1$$

Donde el $-1$ sirve para quitar el caso donde el coeficiente de grado $n$ es 0. Vemos pues que:
$$f(n)=2^{n}-1<2^n, \forall n\in\N$$

Por lo que la mejor aproximación es $f(n)=O(2^n)$\\

7. El n\'umero de polinomios en $x$ de grado a lo m\'as $n-1$ cuyos coeficientes son enteros entre 0 o n.\\
Usando un m\'etodo parecido al del inciso anterior, se puede hacer:
$$n\cdot n\cdot...\cdot n-1$$

Donde, nuevamente, se multiplican $n-1$ veces $n$ y se resta uno para quitar el caso en que el coeficiente de grado $n-1$ sea 0. Así pues, tenemos 
$$f(n)=n^{n-1}-1<n^{n-1}<n^n$$

Por lo que la mejor aproximaci\'on es $f(n)=O(n^n)$\\

8. Dado $k\in\{0,1,2,n-1\}$, la suma $\sum_{i=1}^{n}{i^k}$.\\
Nuevamente, tenemos que a lo mucho, esta suma ser\'a de la forma:
$$f(n)=\sum_{i=1}^{n}{i^{n-1}}=1^{n-1}+2^{n-1}+3^{n-1}+...+n^{n-1}<n^{n-1}+n^{n-1}+n^{n-1}+...+n^{n-1}=n(n^{n-1})=n^n$$
Por lo que la mejor aproximaci\'on es $f(n)=O(n^n)$\\

9. La cantidad de espacio de memoria que una computadora requiere para almacenar un n\'umero $n$.\\
La mejor aproximaci\'on es $f(n)=O(ln(n))$\\


10. La cantidad de espacio de memoria que una computadora requiere para almacenar $n^2$.\\
La mejor aproximaci\'on es $f(n)=O(ln^2(n))$\\

11. La suma de los primeros $n$ enteros positivos.\\
Consideremos que la suma de los primeros n enteros está dada por:
$$f(n)=\frac{n(n+1)}{2}=\frac{n^2+n}{2}\leq n^2$$
Lo \'ultimo se puede ver porque:
$$2n^2\geq n^2+n\Longrightarrow n^2\geq \frac{n^2+n}{2}$$
Por lo que la mejor aproximaci\'on es $f(n)=O(n^2)$\\

12. La suma de los cuadrados de los primeros $n$ enteros positivos.\\
$$f(n)=\frac{n(n+1)(2n+1)}{6}=\frac{n^3}{3}+\frac{n^2}{2}+\frac{n}{6}$$

Así pues, vemos que cada uno de los sumando est\'a acotado por $n^3$. Así pues, la mejor aproximaci\'on es $f(n)=O(n^3)$\\

13. El n\'umero de bits en la suma de los cuadrados de los primeros $n$ enteros positivos.\\
La mejor aproximaci\'on es $f(n)=O(ln^3(n))$\\

\end{document}