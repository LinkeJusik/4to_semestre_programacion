\documentclass[12pt]{article}

\usepackage{hyperref}

\usepackage[spanish]{babel}
\usepackage[letterpaper]{geometry}
\usepackage{amsmath}
\usepackage{amssymb}
\usepackage{amsfonts}

\author{Mart\'{\i}nez Gonz\'alez, Gabriel}
\date{\today}
\title{Tarea 2: Notas Prog II}

\begin{document}

\maketitle

\begin{abstract}

	Genere un documento que tenga dos secciones, una de nombre Lenguaje C y otra de nombre Lenguaje Java. En cada una de las secciones, genere cuatro subsescciones de nombre Palabras reservadas, Tipos de datos b\'asicos o Primitivos, Operadores y Entrada y salida est\'andar. En la subsecci\'on de tipos de datos, genere las subsubsecciones que correspondan a cada uno de los grupos de tipos, como por ejemplo n\'umeros enteros y n\'umeros de punto flotante, en la de operadores, genere las subsubsecciones que correspondan a los tipos de operadores, como por ejemplo aritm\'eticos, l\'ogicos, etc.

\end{abstract}

\section{Lenguaje C}

\subsection{Palabras reservadas}

En el lenguaje C exiten cerca de 40 palabras reservadas, las cuales son:

\begin{center}
\begin{tabular}{lcccccr}

auto & {break} & case & char & const & continue & default \\
do & double & else & enum & extern & float & for \\
goto & if & int & long & register & return & short \\
signed & sizeof & static & struct & typedef & union & unsigned \\
void & volatile & while &  &  &  &  \\

\end{tabular}
\end{center}

\subsection{Tipos de datos b\'asicos o primitivos}

En el lenguaje C existen, de primeras, 6 tipos de datos:

\subsubsection{Car\'acteres}

Se denotan con la palabra reservada \textit{char}. Poseen un tama\~no de un byte (8 bits) y puede ser modificado mendiante modificadores de signo (signed y unsigned).

\subsubsection{N\'umeros enteros}

Se denotan por la palabra reservada \textit{int}. Poseen un tama\~no de 4 bytes y pueden ser modificados por modificadores de tamaño (long o short) o por modificadores de signo (signed o unsigned). 

\subsubsection{N\'umeros de punto flotante}

Se denotan por la palabra reservada \textit{float}. Posee un tama\~no de 4 bytes.

\subsubsection{N\'umero de punto flotante de doble precisi\'on}

Se denotan por la palabra reservada \textit{double}. Poseen un tama\~no de 8 bytes y pueden ser modificador por el modificador \textit{long}.

\subsubsection{Nulo o vacio}

Se denota por la palabra reservada \textit{void} y no debe ocupar espacio en memoria.

\subsubsection{Apuntadores}

Se denota por el simbolo \textit{\&} y ocupa un espacio en memoria de 4 bytes.

\subsection{Operadores}

\subsubsection{Aritm\'eticos}

\begin{itemize}
\item + (suma)
\item - (resta)
\item * (producto)
\item / (divisi\'on)
\item \% (m\'odulo)
\end{itemize}

\subsubsection{Relacionales}

\begin{itemize}
\item < (menor que)
\item > (mayor que)
\item <= (menor o igual que)
\item >= (mayor o igual que)
\item == (igual a)
\item \!= (diferente de)
\end{itemize}

\subsubsection{L\'ogicos}
\begin{itemize}
\item \&\& (and)
\item || (or)
\end{itemize}

\subsubsection{Bit a bit}
\begin{itemize}
\item \& (and)
\item | (or)
\item \! (no)
\end{itemize}

\subsubsection{Asignaci\'on}
\begin{itemize}
\item +=
\item -=
\item =
\item *=
\item /=
\item \%=
\end{itemize}

\subsection{Entrada y salida est\'andar}

Se pueden considerar como entrada y salida est\'andar aquellas que ofrece la biblioteca stdio.h; es decir, utilizando las funciones \textit{scanf} y \textit{printf}.

\section{Lenguaje Java}

\subsection{Palabras reservadas}

\begin{center}
\begin{tabular}{lcccr}

abstract & continue & for & new & switch \\
assert & default & goro & package & synchronized \\
boolean & do & if & private & this \\
break & double & implements & protected & throw \\
byte & else & import & public & throws \\
case & enum & instanceof & return & transient \\
catch & extends & int & short & try \\
char & final & interface & static & void \\
class & finally & long & strictfp & volatile \\
const & float & native & super & while \\

\end{tabular}
\end{center}

\subsection{Tipos de datos b\'asicos o primitivos}

\subsubsection{Booleanos}

Denotados por la palabra \textit{boolean}, ocupan un tama\~no de un bit y se utilizan para denotar la verdad o falsedad de alguna sentencia.

\subsubsection{Car\'acteres}
Denotando por la palabra \textit{char}, ocupa un tama\~no en memoria de 2 bytes.

\subsubsection{N\'umeros enteros}
Denotados por la palabra \textit{int}, ocupan un tama\~no en memoria de 4 bytes; sin embargo, este puede modificarse, o bien definiendo el entero como \textit{byte} (entero de 8 bits); o bien con los modificadores de tamaño \textit{short} o \textit{long}.

\subsubsection{N\'umeros con punto flotante}
Denotados por la palabra \textit{float}, ocupan un tama\~no en memoria de 4 bytes.

\subsubsection{N\'umeros con punto flotante de doble precisi\'on}
Denotados por la palabra \textit{double}, ocupan un tama\~no en memoria de 8 bytes.

\subsection{Operadores}

\subsubsection{Aritm\'eticos}

\begin{itemize}
\item + (suma)
\item - (resta)
\item * (producto)
\item / (divisi\'on)
\item \% (m\'odulo)
\end{itemize}

\subsubsection{Relacionales}

\begin{itemize}
\item < (menor que)
\item > (mayor que)
\item <= (menor o igual que)
\item >= (mayor o igual que)
\item == (igual a)
\item \!= (diferente de)
\end{itemize}

\subsubsection{L\'ogicos}
\begin{itemize}
\item \&\& (and)
\item || (or)
\end{itemize}

\subsubsection{Concatenaci\'on}
\itemize{+}

\subsubsection{Bit a bit}

\begin{itemize}
\item \& (and)
\item | (or)
\item \<\< (desplazamiento a la izquierda de los bits del operando)
\item \>\> (desplazamiento a la derecha de los bits del operando)
\item \! (no)
\end{itemize}

\subsubsection{Asignaci\'on}

\begin{itemize}
\item +=
\item -=
\item =
\item *=
\item /=
\item \%=
\end{itemize}

\subsection{Entrada y salida est\'andar}
Java tiene acceso a la entrada y salida estandar a trav\'es de la clase \textit{System}. En concreto, atravez de los ficheros:
\begin{itemize}
\item Stdin: \textit{System.in} implementa stdin como una instancia de la clase Input Stream. Con \textit{System.in} se accede a los m\'etodos \textit{read()} y \textit{skip()}. En concreto, \textit{System.in.read()} es la instancia más utilizada para leer datos de entrada de cualquier tipo.
\item Stdout: \textit{System.out} implementa stdout como una instancia de la clase PrintStream. Se pueden utilizar los m\'etodos \textit{print()} y \textit{println()} con cualquier tipo de dato b\'asico de Java como argumento.
\end{itemize}

\nocite{*}
\bibliography{biblio} \bibliographystyle{alpha}

\end{document}
