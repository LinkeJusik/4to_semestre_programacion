%
% Carlos Coronado
% Materia: Programaci'on II
%

\documentclass{beamer}  % Tipo de documento, en este caso: presentaci'on

\mode<presentation>
{
  \usetheme{% Tema de la presentaci'on (colores, formas, etc.)
    %AnnArbor%
    %Antibes%
    %Bergen%
    %Berkeley%
    %Berlin%
    %Boadilla%
    %boxes%
    %CambridgeUS%
    %Copenhagen%
    Darmstadt%
    %default%
    %Dresden%
    %Frankfurt%
    %Goettingen%
    %Hannover%
    %Ilmenau%
    %JuanLesPins%
    %Luebeck%
    %Madrid%
    %Malmoe%
    %Marburg%
    %Montpellier%
    %PaloAlto%
    %Pittsburgh%
    %Rochester%
    %Singapore%
    %Szeged%
    %Warsaw%
  }

  \setbeamercovered{transparent} % Opcional para la aparicion gradual
  \usecolortheme[rgb={0.32,0.67,0.39}]{structure} % colores con intensidad
  %\usecolortheme[rgb={0.55,0,0.15}]{structure} % colores con intensidad
                                               % del rojo, verde y azul
                                               % valores entre 0 y 1
                                               % pruebe: 0, 0, 1
                                               % pruebe: 0, 1, 0
                                               % pruebe: 1, 0, 0
                                               % pruebe: 0.32, 0.67, 0.39
}


\usepackage[english]{babel} % el paquete spanish puede tener ciertos conflictos



\title{T\'{\i}tulo de la Presentaci\'on}

\subtitle{Subt\'{\i}tulo de la Presentaci\'on} % (optional)

\author[Coronado] % (optional. Nombre ``corto'')
{Carlos Coronado}

\institute[ESFM] % (optional. Nombre ``corto'')
{
  Escuela Superior de F\'{\i}sica y Matem\'aticas\\
  Instituto Polit\'ecnico Nacional}

\date[dd/mm/aa] % (optional)
{d\'{\i}a de mes de a\~no}


% para que se vea el contenido al paso de cada secci#on
\AtBeginSubsection[]
{
  \begin{frame}<beamer>
    \frametitle{Contenido}
    \tableofcontents[currentsection,currentsubsection]
  \end{frame}
}

%%%% definici'on de algunos ambientes

\newtheorem{miTeo}{Teorema}
\newtheorem{miDef}{Definici\'on}

\begin{document}

\begin{frame}
  \titlepage
\end{frame}

\begin{frame}
  \frametitle{Contenido}
  \tableofcontents
\end{frame}


\section{Introducci\'on}

\subsection[1$^{\mbox{\tiny ra}}$ subsecci\'on]{Nombre de la Primera Subsecci\'on}

\begin{frame}
  \frametitle{T\'{\i}tulos Informativos. Uso de May\'usculas.}
  \framesubtitle{Subt\'{\i}tulos Optionales.}

  \begin{itemize}[<+->] % para que parezcan uno a uno
  \item Use mucho \texttt{itemize}.
  \item Use sentencias o frases muy breves.
  \end{itemize}

\end{frame}

\subsection[2$^{\mbox{\tiny da}}$ subsecci\'on]{Nombre de la Segunda Subsecci\'on}

\begin{frame}
  \frametitle{T\'{\i}tulos Informativos.}

  Se pueden crear capas sobrepuestas\dots
  \begin{itemize}
  \item usando el comando \texttt{pause}:
    \begin{itemize}
    \item Primer item.  \pause
    \item Segundo item.
    \end{itemize}
  \item usando especificaciones sobrepuestas:
    \begin{itemize}
    \item<3-> Primer item.
    \item<4-> Segundo item.
    \end{itemize}
  \item usando el comando general \texttt{uncover}:
    \begin{itemize}
      \uncover<5->{\item Primer item.}
      \uncover<6->{\item Segundo item.}
    \end{itemize}
  \end{itemize}
\end{frame}


\section{Segunda Secci\'on}

\begin{frame}
  \frametitle{T\'{\i}tulos Informativos.}

  \begin{miTeo}[Fundamental de la Aritm\'etica]

    Todo n\'umero entero se puede factorizar como producto de primos.
    La factorizaci\'on es \'unica salvo orden y unidades.

  \end{miTeo}

\end{frame}

\begin{frame}
  \frametitle{T\'{\i}tulos Informativos.}

  \begin{miDef}
    Sea $\{a_n\}_{n=1}^\infty$ una sucesi\'on, decimos que $l$ es el
    l\'{\i}mite de $\{a_n\}_{n=1}^\infty$ si $\forall\ \epsilon>0
    \quad \exists n_0 \geq 1$ tal que $|a_n - l| < \epsilon \quad
    \forall n\geq n_0$.

  \end{miDef}

\end{frame}


\section*{Resumen}

\begin{frame}
  \frametitle{Resumen}

  % Mantenga el resumen muy ``breve''.
  \begin{itemize}
  \item El \alert{primer mensaje principal} de la pl\'atica en una o
    dos l\'{\i}neas.
  \item El \alert{segundo mensaje principal} de la pl\'atica en una o
    dos l\'{\i}neas.
  \item Quiz\'a un \alert{Tercer mensaje}, pero no m\'as que eso.
  \end{itemize}
  
\end{frame}


\end{document}


