%
% Carlos Coronado
% Materia: Programaci'on II
%

\documentclass{article}            % Tipo de documento, en este caso: art'iculo

\usepackage[spanish]{babel}        % algunas macros para signos de puntuaci'on en espa~nol
\usepackage[letterpaper]{geometry} % tipo de hoja (tama~no carta)
\usepackage{amsmath}               % \
\usepackage{amssymb}               %  > s'imbolos, fuentes, etc. de la AMS
\usepackage{amsfonts}              % /  (s'imbolos matem'aticos)
\newcommand{\C}{\ensuremath{\mathbb{C}}} % C como de numeros Complejos
\newcommand{\R}{\ensuremath{\mathbb{R}}} % R
\newcommand{\Z}{\ensuremath{\mathbb{Z}}} % Z
\newcommand{\N}{\ensuremath{\mathbb{N}}} % N

\author{Carlos Coronado}
\title{Ejemplo con Varios Archivos}
\date{\today}

\begin{document}                   % inicio de documento

\maketitle                         % se genera un t'itulo del documento

\begin{abstract}                   % describir brevemente de que se trata 
                                   %el documento

  En este documento se muestran algunos ejemplos de los ambientes
  \texttt{tabbing}, \texttt{table}, \texttt{figure} y
  \texttt{verbatim}.  El contenido del documento est\'a distribuido en
  varios archivos.

\end{abstract}

\section{Introducci\'on}

  Se mostrar\'a algunos ambientes y comandos de \LaTeX.

%
% Carlos Coronado
% Materia: Programaci'on II
%

\section{Ambiente \texttt{table}}

Para crear un cuadro e ir llevando la cuenta de estos elementos de
forma autom\'atica, utilizamos el ambiente \texttt{table} como se
muestra a continuaci\'on. Ver cuadro \ref{mi:cuadro} en la p\'agina
\pageref{mi:cuadro}.

\begin{table}
  \begin{center}

    Es importante utilizar etiquetas para hacer referencia a este
    cuadro.

    Se puede hacer uso del ambiente \texttt{tabular} aqu\'{\i}
    adentro, por ejemplo.

    \caption{Dentro del ambiente}\label{mi:cuadro}

  \end{center}
\end{table}

%
% Carlos Coronado
% Materia: Programaci'on II
%

\section{Ambiente \texttt{figure}}

Para crear una figura e ir llevando la cuenta de estos elementos de
forma autom\'atica, utilizamos el ambiente \texttt{figure} como se
muestra a continuaci\'on. Ver figura \ref{mi:figura}\footnote{el
  comando \texttt{includegraphics} y otras formas de insertar
  im\'agenes se ver\'an m\'as adelante} en la p\'agina
\pageref{mi:figura}

\begin{figure}

  \begin{center}
    Es importante utilizar etiquetas para hacer referencia a esta
    figura.

    Se puede hacer uso del comando \texttt{includegraphics} aqu\'{\i}
    adentro, por ejemplo.

    \caption{Dentro del ambiente}\label{mi:figura}

  \end{center}
\end{figure}

%
% Carlos Coronado
% Materia: Programaci'on II
%

\section{Ambiente \texttt{verbatim}}

Para escribir informaci\'on tal cual esta en texto y sin ningun
formato, utilizamos el ambiente \texttt{verbatim} como se muestra a
continuaci\'on.

\begin{verbatim}

Lo que se escriba a qu\'{\i} va a aparecer tal cual.

  con muchos o pocos         espacios        y otras

         elementos ...


          _ _ _ _ _ _ _ _ 
\end{verbatim}


\end{document}
