%
% Carlos Coronado
% Materia: Programaci'on II
%

\documentclass{article}            % Tipo de documento, en este caso: art'iculo

\usepackage{hyperref}

\usepackage[spanish]{babel}        % algunas macros para signos de puntuaci'on en espa~nol
\usepackage[letterpaper]{geometry} % tipo de hoja (tama~no carta)
\usepackage{amsmath}               % \
\usepackage{amssymb}               %  > s'imbolos, fuentes, etc. de la AMS
\usepackage{amsfonts}              % /  (s'imbolos matem'aticos)
\newcommand{\C}{\ensuremath{\mathbb{C}}} % C como de numeros Complejos
\newcommand{\R}{\ensuremath{\mathbb{R}}} % R
\newcommand{\Z}{\ensuremath{\mathbb{Z}}} % Z
\newcommand{\N}{\ensuremath{\mathbb{N}}} % N


\author{Carlos Coronado}
\title{Ejemplo para Inclusi\'on de Bibliograf\'{\i}a}
\date{\today}

\begin{document}                   % inicio de documento

\maketitle                         % se genera un t'itulo del documento

\begin{abstract}                   % describir brevemente de que se trata 
                                   %el documento

  En este documento se muestra un ejemplo para incluir
  bibliograf\'{\i}a en un documento \LaTeX.

\end{abstract}

En el libro de Knuth \cite{knuth} se muestran diversos algoritmos
computacionales.  En el reporte t\'ecnico \cite{zimmermann} se muestra
un m\'etodo para la multiplicaci\'on de n\'umeros enteros de longitud
``grande''.  En el art\'{\i}culo \cite{rsa}, los autores presentan un
esquema criptogr\'afico de clave p\'ublica para el uso de firmas digitales.
 
En la bibliograf\'{\i}a puede aparecer uno o m\'as elementos
bibliogr\'aficos que no se hayan citado expl\'{\i}citamente, como puede verse en este ejemplo al utilizar \verb|\nocite{bm}|, como ejemplo.

 \nocite{bm,hashmd5}

\bibliography{miBiblio} \bibliographystyle{alpha}


\end{document}
