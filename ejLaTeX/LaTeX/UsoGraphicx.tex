%
% Carlos Coronado
% Materia: Programaci'on II
% 

\documentclass{article}            % Tipo de documento, en este caso: art'iculo

\usepackage[spanish]{babel}        % algunas macros para signos de puntuaci'on en espa~nol
\usepackage[letterpaper]{geometry} % tipo de hoja
\usepackage{amsmath}               % \
\usepackage{amssymb}               %  > s'imbolos, fuentes, etc. de la AMS
\usepackage{amsfonts}              % /  (s'imbolos matem'aticos)
\newcommand{\C}{\ensuremath{\mathbb{C}}} % C como de numeros Complejos
\newcommand{\R}{\ensuremath{\mathbb{R}}} % R
\newcommand{\Z}{\ensuremath{\mathbb{Z}}} % Z
\newcommand{\N}{\ensuremath{\mathbb{N}}} % N

%%%%%%%%%%%%%%%%%%%%%%%%%%%%%%%%%%%%
\usepackage{graphicx}              % un modo de incluir im'agenes
%%%%%%%%%%%%%%%%%%%%%%%%%%%%%%%%%%%%


\author{Carlos Coronado}
\title{Ejemplo para Inclusi\'on de Archivos de Im\'agenes}
\date{\today}

\begin{document}                   % inicio de documento

\maketitle                         % se genera un t'itulo del documento

\begin{abstract}                   % describir brevemente de que se trata 
                                   %el documento

  En este documento se muestra un ejemplo para incluir archivos con
  im\'agenes.

\end{abstract}


  Con el comando
  \verb|\includegraphics[|\textit{par\'ametros}\verb|]{|\textit{archivo}\verb|}|
  se puede incluir una imagen como \includegraphics[scale=0.5]{esfm}, 
  o como

  \begin{center}
  \includegraphics[scale=0.5]{ipn}
  \end{center}

  Tambi\'en se puede usar el entorno \texttt{figure}

\begin{figure}
  \begin{center}
    \includegraphics{esfm} 

    \caption{Figura de un archivo en formato png, otros tambi\'en
    aceptados son: jpg y pdf}\label{miFIG}

  \end{center}
\end{figure}

En la figura \ref{miFIG} se muestra un ejemplo del entorno
\texttt{figure}.


\end{document}
