%
% Carlos Coronado
% Materia: Programaci'on II
%

\documentclass{article}            % Tipo de documento, en este caso: art'iculo

\usepackage[spanish]{babel}        % algunas macros para signos de puntuaci'on en espa~nol
\usepackage[letterpaper]{geometry} % tipo de hoja (tama~no carta)
\usepackage{amsmath}               % \
\usepackage{amssymb}               %  > s'imbolos, fuentes, etc. de la AMS
\usepackage{amsfonts}              % /  (s'imbolos matem'aticos)
\newcommand{\C}{\ensuremath{\mathbb{C}}} % C como de numeros Complejos
\newcommand{\R}{\ensuremath{\mathbb{R}}} % R
\newcommand{\Z}{\ensuremath{\mathbb{Z}}} % Z
\newcommand{\N}{\ensuremath{\mathbb{N}}} % N

%%%%%%%%%%%%%%%%%%%%%%%%%%%%%%%%%%%%
\usepackage{makeidx}               % un paquete para incluir 'indice
%%%%%%%%%%%%%%%%%%%%%%%%%%%%%%%%%%%% (referencia de texto en el doc)
\makeindex                         % importante para poder generar el 
                                   % 'indice (mas no la tabla de contenido)

\author{Carlos Coronado}
\title{Ejemplo para Inclusi\'on de Contenido, Listados (de Tablas y Figuras) 
e \'Indice}
\date{\today}

\begin{document}                   % inicio de documento

\maketitle                         % se genera un t'itulo del documento

\tableofcontents         % se genera la tabla de contenidos
\listoffigures           % se genera la lista de figuras (ambiente figure)
\listoftables            % se genera la lista de tablas (ambiente table)

\begin{abstract}                   % describir brevemente de que se trata 
                                   %el documento

  En este documento se muestra un ejemplo para incluir listado de
  tablas, figuras y contenido, as\'{\i} como un \'{\i}ndice.

\end{abstract}

\section{Introducci\'on}

  Para incluir el contenido de un documento \LaTeX, en cuanto a las
  secciones, subsecciones, etc., se refiere, se utiliza el comando
  \verb|\tableofcontents|.  Para enlistar las tablas y/o figuras
  contenidas en el documento se utiliza el comando
  \verb|\listoftables| y \verb|\listoffigures|, respectivamente.

\section{Figuras y Tablas}

  Las figuras\index{figura} y tablas\index{tabla} hacen referencia al
  ambiente \texttt{figure} y \texttt{table}, respectivamente.

\subsection{Figuras}

  Una figura se puede incluir con el ambiente \texttt{figure}.

  \begin{figure}

   Aqu\'{\i} deber\'{\i}a ir una figura.

  \caption{Ejemplo del ambiente \texttt{figure}}\label{fig:ejemplo}


  \end{figure}

  En la figura \ref{fig:ejemplo}\index{figura!ejemplo} se muestra un
  ejemplo del uso del ambiente \texttt{figure}.

\subsection{Tablas}

  Una tabla\index{tabla} se puede incluir con el ambiente \texttt{table}.

  \begin{table}

   Aqu\'{\i} deber\'{\i}a ir una tabla.

  \caption{Ejemplo del ambiente \texttt{table}}\label{tab:ejemplo}

  \end{table}

  En la tabla \ref{tab:ejemplo}\index{tabla!ejemplo} se muestra un
  ejemplo del uso del ambiente \texttt{table}.

\section{Final}

  Mera secci\'on de relleno\index{relleno} para el uso de la tabla de
  contenidos.


  \index{ejemplos|see{figura}}
  \index{ejemplos|see{tabla}}

  \printindex

%%%%
%%%% IMPORTANTE:
%%%%    Al compilar el documento LaTeX, se debe ejecutar el comando
%%%% makeindex (ESTO LO DEBER'IA HACER EL ENTORNO DE DESARROLLO) para
%%%% poder generar el 'indice.
%%%%

\end{document}
