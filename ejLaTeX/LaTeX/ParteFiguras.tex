%
% Carlos Coronado
% Materia: Programaci'on II
%

\section{Ambiente \texttt{figure}}

Para crear una figura e ir llevando la cuenta de estos elementos de
forma autom\'atica, utilizamos el ambiente \texttt{figure} como se
muestra a continuaci\'on. Ver figura \ref{mi:figura}\footnote{el
  comando \texttt{includegraphics} y otras formas de insertar
  im\'agenes se ver\'an m\'as adelante} en la p\'agina
\pageref{mi:figura}

\begin{figure}

  \begin{center}
    Es importante utilizar etiquetas para hacer referencia a esta
    figura.

    Se puede hacer uso del comando \texttt{includegraphics} aqu\'{\i}
    adentro, por ejemplo.

    \caption{Dentro del ambiente}\label{mi:figura}

  \end{center}
\end{figure}
