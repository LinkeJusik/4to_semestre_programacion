%
% Carlos Coronado
% Materia: Programaci'on II
%

\documentclass[12pt]{article}

\usepackage[spanish]{babel}
\usepackage[letterpaper]{geometry}

\author{Nombre Alumno(a)}
\date{\today}
\title{T\'{\i}tulo de la Tarea o Trabajo}

\begin{document}

\maketitle

\begin{abstract}

  Breve descripci\'on de lo abarcado en la tarea o el trabajo.

\end{abstract}

\section{Nombre de una secci\'on}

Informaci\'on sobre la secci\'on.

\subsection{Nombre de una subsecci\'on}

Posible subsecci\'on o tema a tratar dentro de la presente secci\'on.

\subsection{Signos de puntuaci\'on}

Acentos: \'a, \'e, \'{\i}, \'o, \'u.

\~n y otros signos ?`, !` se producen con:

\begin{verbatim}
Acentos: \'a, \'e, \'{\i}, \'o, \'u.

\~n y otros signos ?`, !` se producen con:
\end{verbatim}

\section{Otra secci\'on}

Informacion sobre esta otra secci\'on.

\section{Ambiente Matem\'atico}

Es esta secci\'on se describe muy brevemente el ambiente o modo
matem\'atico.

Para escribir f\'ormulas dentro de un texto se debe utilizar comandos
matem\'aticos dentro de \$ \$. Por ejemplo $\frac{\int_0^\infty
  e^{-x^2}dx}{\pi}$.

Para escribir una f\'ormula matem\'atica, tambi\'en se puede utilizar
dobles signos de pesos (\$\$) al inicio y final de la f\'ormula, lo la
centra. Por ejemplo la misma expresi\'on da como resultado
$$\frac{\int_0^\infty e^{-x^2}dx}{\pi}$$

\section{Tablas}

En \LaTeX\ se pueden utilizar tablas como por ejemplo:

\begin{center}

\begin{tabular}{lcr}

casilla izq & casilla central & casilla der \\
otra & otra & otra \\
una m\'as & una m\'as & una m\'as 

\end{tabular}

\end{center}

o como:

\begin{center}

\begin{tabular}{||l|c|r||}

\hline
casilla izq & casilla central & casilla der \\ \hline
otra & otra & otra \\ \hline
una m\'as & una m\'as & una m\'as \\ \hline

\end{tabular}

\end{center}

La tabla tambien se puede poner dentro de una ``lista de cuadros''
como por ejemplo:

\begin{table}

\begin{tabular}{||l|c|r||}

\hline
casilla izq & casilla central & casilla der \\ \hline
otra & otra & otra \\ \hline
una m\'as & una m\'as & una m\'as \\ \hline

\end{tabular}

\caption{Ejemplo de una Tabla}\label{miTabla}

\end{table}

En la tabla \ref{miTabla} se muestra un ejemplo de informaci\'on en
\emph{recuadros}.



\end{document}
