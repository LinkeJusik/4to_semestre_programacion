%
% Carlos Coronado
% Materia: Programaci'on II
%
\documentclass{article}

\newtheorem{teo}{Teorema}

\newcommand{\sen}{\ensuremath{\mathrm{sen}}}

\begin{document}

Ejemplo de ecuaciones matem\'aticas (uso del modo matem\'atico).

\begin{equation}
 a^2 + b^2 = c^2\label{TeoPitagoras}
\end{equation}

En la ecuaci\'on \ref{TeoPitagoras} se muestra el teorema de
Pit\'agoras, en donde $a$ y $b$ son los catetos de un tri\'angulo
rect\'angulo y $c$ es su hipotenusa.

\begin{teo}[Fundamental de la Aritm\'etica]

  Todo n\'umero entero se factoriza en producto de primos de forma
  \'unica salvo orden y unidades.

\end{teo}

Ejemplo de l\'{\i}mites: $\lim_{x \rightarrow 0} \frac{\sen x}{x}=1$,
o se puede escribir as\'{\i} $$\lim_{x \rightarrow 0} \frac{\sen
  x}{x}=1$$ o de esta otra forma:

\[\lim_{x \rightarrow 0} \frac{\sen x}{x}=1, \sum_{i=0}^n a_i\]


\end{document} 
 
