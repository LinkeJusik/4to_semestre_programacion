%
% Carlos Coronado
% Materia: Programaci'on II
%

\documentclass{beamer}  % Tipo de documento, en este caso: presentaci'on

\mode<presentation>
{
  \usetheme{% Tema de la presentaci'on (colores, formas, etc.)
    AnnArbor%
    %Antibes%
    %Bergen%
    %Berkeley%
    %Berlin%
    %Boadilla%
    %boxes%
    %CambridgeUS%
    %Copenhagen%
    %Darmstadt%
    %default%
    %Dresden%
    %Frankfurt%
    %Goettingen%
    %Hannover%
    %Ilmenau%
    %JuanLesPins%
    %Luebeck%
    %Madrid%
    %Malmoe%
    %Marburg%
    %Montpellier%
    %PaloAlto%
    %Pittsburgh%
    %Rochester%
    %Singapore%
    %Szeged%
    %Warsaw%
  }

  \setbeamercovered{transparent} % Opcional para la aparicion gradual
}


\usepackage[english]{babel} % el paquete spanish puede tener ciertos conflictos



\title{T\'{\i}tulo de la Presentaci\'on}

\subtitle{Subt\'{\i}tulo de la Presentaci\'on} % (optional)

\author[Coronado] % (optional. Nombre ``corto'')
{Carlos Coronado}

\institute[ESFM] % (optional. Nombre ``corto'')
{
  Escuela Superior de F\'{\i}sica y Matem\'aticas\\
  Instituto Polit\'ecnico Nacional}

\date[dd/mm/aa] % (optional)
{dd de mes de aaaa \\o usar el comando $\mathtt{\backslash}$\texttt{today}=(\today)}


%\beamerdefaultoverlayspecification{<+->} % Por si se quiere que aparezca
                                          % gradualmente los items


\begin{document}

\begin{frame}
  \titlepage
\end{frame}

\begin{frame}
  \frametitle{Outline}
  \tableofcontents
\end{frame}


\section{Introducci\'on}

\subsection[1$^{\mbox{\tiny ra}}$ subsecci\'on]{Nombre de la Primera Subsecci\'on}

\begin{frame}
  \frametitle{T\'{\i}tulos Informativos. Uso de May\'usculas.}
  \framesubtitle{Subt\'{\i}tulos Optionales.}

  \begin{itemize}[<+->] % para que parezcan uno a uno
  \item Use mucho \texttt{itemize}.
  \item Use sentencias o frases muy breves.
  \end{itemize}

\end{frame}

\begin{frame}
  \frametitle{T\'{\i}tulos Informativos.}

  Se pueden crear capas sobrepuestas\dots
  \begin{itemize}
  \item usando el comando \texttt{pause}:
    \begin{itemize}
    \item Primer item.  \pause
    \item Segundo item.
    \end{itemize}
  \item usando especificaciones sobrepuestas:
    \begin{itemize}
    \item<3-> Primer item.
    \item<4-> Segundo item.
    \end{itemize}
  \item usando el comando general \texttt{uncover}:
    \begin{itemize}
      \uncover<5->{\item Primer item.}
      \uncover<6->{\item Segundo item.}
    \end{itemize}
  \end{itemize}
\end{frame}


\subsection{Segunda Subsecci\'on}

\begin{frame}
  \frametitle{T\'{\i}tulos Informativos.}
\end{frame}

\begin{frame}
  \frametitle{T\'{\i}tulos Informativos.}
\end{frame}


\section*{Resumen}

\begin{frame}
  \frametitle{Resumen}

  % Mantenga el resumen muy ``breve''.
  \begin{itemize}
  \item El \alert{primer mensaje principal} de la pl\'atica en una o
    dos l\'{\i}neas.
  \item El \alert{segundo mensaje principal} de la pl\'atica en una o
    dos l\'{\i}neas.
  \item Quiz\'a un \alert{Tercer mensaje}, pero no m\'as que eso.
  \end{itemize}
  
\end{frame}


\end{document}


